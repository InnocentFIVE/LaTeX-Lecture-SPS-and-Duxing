\section{安装}

\begin{frame}[standout]
	真的要安装吗?

	\textcolor[HTML]{42AC47}{overleaf} 与 \textcolor[HTML]{A93529}{ShareLaTeX}

	\includegraphics[width=.7\textwidth]{images/ol_plus_sl.png}

	\pause\small\mdseries 对于免费版使用者的限制见 \link{https://www.overleaf.com/learn/how-to/Overleaf_plan_limits}(比如$\qty{20}{s}$编译时间?)
	\nonumberfootnote{红配绿,那啥……}
\end{frame}

\begin{frame}{或者}
	\begin{itemize}
		\item<+-> 越来越多的用户正在使用的 TeXPage(对中文的支持更好些,产品定价 \link{https://www.texpage.com/pricing}):
			\begin{center}
				\includegraphics[width=.8\textwidth]{images/editor-users.png}
			\end{center}
		\item<+-> 合理的建议:先不要急着安装,用用在线平台试试手;
		\item<+-> 以上这些云端部署的情形可以免去安装、升级等一系列烦恼,更方便多人协作,如果你是本地人的话……
	\end{itemize}
\end{frame}

\begin{frame}{请选择你的发行版}
	\begin{itemize}
		\item 发行版是引擎$+$宏包$+$字体$+$文档$+$各种各样的东西……
		\item 也就是“全家桶”.
	\end{itemize}
	\medskip
	\begin{columns}[t]
		\pause
		\begin{column}{.3\textwidth}
			\huge\TeX{}~live\small
			\begin{itemize}
				\item \faWindows~\faApple~\faLinux
				\item 官方维护,居家必备;
				\item 不方便出门携带(指安装包体积很大);
				\item 每年都要更新.
			\end{itemize}
		\end{column}
		\pause
		\begin{column}{.3\textwidth}
			\huge MiK\TeX\small
			\begin{itemize}
				\item \faWindows~\faApple~\faLinux
				\item 宏包可以等到要用的时候再装;
				\item 安装宏包时可能卡网络问题.
			\end{itemize}
		\end{column}
		\pause
		\begin{column}{.35\textwidth}
			\huge \CTeX\small
			\begin{itemize}
				\item \faWindows
				\item 2022年,吴凌云提交了 \CTeX{} 的更新,所以现在也许能用,但暂不推荐;
				\item 处理历史文档或投稿部分国内期刊时可以考虑使用.
			\end{itemize}
		\end{column}
	\end{columns}

	\pause\tiny Tiny\TeX{}:字面意思上地,是\TeX{}~live的微缩版,假设用户不惧怕或反感使用命令行.
\end{frame}


\begin{frame}{专武}
    \DifficultyOfEditor{2}{Easy}{TeXWorks \& \TeX{}~Studio}{开箱即用,专为\TeX{}使用者定制.}\pause
    \DifficultyOfEditor{4}{Normal}{Visual Studio Code}{最新最热的代码编辑器(之一).}\pause
    \DifficultyOfEditor{6}{Hard}{Vim}{获得乱码的最好方式是打开Vim直到保存退出.}\pause
    \DifficultyOfEditor{8}{Lunatic}{针}{小心出国杳无音讯.}
	\nonumberfootnote{编辑器全明星大战:\link{https://en.wikipedia.org/wiki/Comparison_of_TeX_editors}}
\end{frame}

\begin{frame}{如何安装?}
	\begin{itemize}
		\item 就交给实操演示吧!\pause
		      \begin{itemize}
			      \item 没有发簪?可以看看清华(\link{https://mirrors.tuna.tsinghua.edu.cn/})、上交(\link{https://mirrors.sjtug.sjtu.edu.cn/})和中科大(\link{https://mirrors.tuna.tsinghua.edu.cn})的镜像网.
		      \end{itemize}\pause
		\item 或者……不惧怕命令行读者的教程~\link{https://github.com/OsbertWang/install-latex-guide-zh-cn}
	\end{itemize}
	\pause
	\begin{axiom}[安装路径]
		路径尽量不要用中文、空格、特殊符号.
	\end{axiom}
\end{frame}


\begin{frame}{开箱}
	\begin{theorem}[命令行]
		很多人都不想看到命令行.
	\end{theorem}
	\pause
	\begin{proof}
		已由~\raisebox{-.3ex}{\simpleicon{cplusplus}}~证明,我们会将命令行的部分放到最后面.需要注意的是,编译的选项很多,远不是一个~\faPlay~能够涵盖的.
	\end{proof}
	\pause
	我们在此暂时忽略易混淆的引擎、格式之类,仅仅列举现今最常用的几种\TeX{}程序对比:\pause
	\begin{itemize}
		\item \pdfLaTeX:不直接支持Unicode,但是支持micro-typography;
		\item \XeLaTeX:支持Unicode和OpenType,目前中文社区的通用程序;
		\item \LuaLaTeX:支持Unicode和OpenType,内联Lua.
	\end{itemize}
\end{frame}


