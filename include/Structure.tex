\sectionwithabstract{文档结构}[%
内容与表现分离(\emph{separation} of content and presentation)原则的本意是「文档的实际内容、逻辑结构」与「文档呈现给读者的样式」是相互独立的.
并且,「内容与样式分离」是一种通用的原则,而并非只属于 \LaTeX{}.%
]

\begin{frame}[fragile]{一个例子}
	\begin{texcode}[basicstyle=\fontsize{7.5}{9}\selectfont\ttfamily, emph={[1]document}, emph={[2]article,amsmath,fixdif}, emph={[12]\Bigl,\Bigr}, emph={[13]\biggl,\biggr}, breaklines = true, moretexcs = {\key}]
\documentclass{article}                 % 导言区. 设置文档样式.
\usepackage{amsmath}
\usepackage{fixdif}                     % 正体微分算子 \d.
\newcommand{\key}[1]{\textbf{#1}}       % 定义新命令.
\begin{document}
    \section{The Spectral Theorem}      % 小节标题.
    If $\mu$ is a \key{%*\textbf{projection-valued measure}*)} on $(X, \Omega)$ with values in $\mathsf B(\mathcal H)$, $\psi$ is an element of $\mathcal H$, then we can construct a positive real-valued measure $\mu_\psi$ from $\mu$ by setting $\mu_\psi(E) = \langle\psi, \mu(E)\psi\rangle$, for each measurable set $E$.

    To motivate the following definition, consider integration of a \emph{%*\textcolor{蔚蓝!50!blue}{\emph{bounded}}*)} measurable function $f$ against a projection-valued measure $\mu$. Since the integral is multiplicative and complex-conjugation of a function corresponds to adjoint of the operator, we have
    \[\biggl\| \Bigl(\int_X f \d \mu\Bigr)\psi \biggr\|^2 = \int_X |f|^2 \d \mu_\psi.\]
\end{document}
\end{texcode}
\end{frame}

\begin{frame}[fragile]{The Spectral Theorem}
	\addfontfeatures{RawFeature=-case}
	If $\mu$ is a \textbf{projection-valued measure} on $(X, \Omega)$ with values in $\symsf B(\mathcal H)$, $\psi$ is an element of $\mathcal H$, then we can construct a positive real-valued measure $\mu_\psi$ from $\mu$ by setting $\mu_\psi(E) = \langle\psi, \mu(E)\psi\rangle$, for each measurable set $E$.

	To motivate the following definition, consider integration of a \emph{bounded} measurable function $f$ against a projection-valued measure $\mu$. Since the integral is multiplicative and complex-conjugation of a function corresponds to adjoint of the operator, we have
	\[
		\biggl\Vert \Bigl(\int_X f \d \mu\Bigr)\psi \biggr\Vert^2 = \int_X \vert f\vert ^2 \d \mu_\psi.
	\]
	\nonumberfootnote{文本来源:\textsc{Brian C. Hall.}~\emph{Quantum Theory for
			Mathematicians}.}
\end{frame}

\begin{frame}[fragile]{布局:从零(一)}
	\LaTeX{} 给了我们这样一个机会去思考一篇文章里面到底有什么,事实上,简单的文章大概都会包含以下这些:\pause
	\begin{itemize}
		\item<+-> 标题.事实上包括标题还有作者的名字,日期,所属机构等等.或者更广泛地,扉页和版权页:|\title|, |\author|, |\date|, |\maketitle|;
		\item<+-> 摘要、致谢与声明.与这些辅文类似的,还有序、前言之类:
			|\begin{abstract}...\end{abstract}|;
		\item<+-> 目录.|\tableofcontents|;
		\item<+-> 你的正文.大致分 |\chapter|, |\section|, |\subsection|, |\paragraph| 等;
		\item<+-> 参考文献.|\bibliography| 与 |\cite|.
	\end{itemize}
\end{frame}

\begin{frame}[fragile]{布局:从零(二)}
	\begin{itemize}
		\item<+-> 在真正写东西之前,了解自己需要写什么:\pkg{article}?\pkg{book}?亦或是只用来做极小测试的 \pkg{minimal}?在想清楚这点之后,我们就可以选取所需的\alert{文档类}了.
		\item<+-> 文档类用 |\documentclass[选项]{名称}| 在引言区调用,三大标准类:
			\begin{itemize}
				\item \pkg{article}:用于科技论文、报告、说明文档等;
				\item \pkg{report}:具有\alert{章节结构},用于综述、长篇论文、简单书籍等;
				\item \pkg{book}:包含章节结构和前言、正文、后记等结构.
			\end{itemize}\onslide<+->
			因此,\pkg{article} 没有定义 |\chapter| 命令合情合理.在科技论文中,如果要将文章分割成多个超大板块,可以用 |\part|.
	\end{itemize}
\end{frame}

\begin{frame}[fragile]{布局:文本标记}
	\textbf{我要加粗}!\emph{我要楷体}!\textsf{我要黑体}\kern\ccwd 和\hfill 右对齐!我还要:
	\centerline{\smash{居中中中中中}}
	\rightline{——{\Large 甲}\small 方}\pause

	\begin{itemize}
		\item<+-> 如果不管格式与内容分离的原则(或者懒),可以用
			\begin{itemize}
				\item 粗体:|\textbf{...}|、|{\bfseries ...}|;
				\item 黑体:|\textsf{...}|、|{\sffamily ...}|;
				\item 意大利体(\emph{斜体}):|\textit{...}|、|{\itshape ...}|;
				\item 控制大小:|\tiny|、|\small|、|\normalsize|、|\large|、|\Huge|.
			\end{itemize}
		\item<+-> 但是为了避免别人诟病,你可以(新定义) |\keyword{...}| 代替 |\textbf{...}|、|\emph{...}| 代替 |\textit{...}|;
		\item<+-> 折行是 |\\|,但这只是折行,分段用 |\par| 或者直接空行;
		\item<+-> 居中可以用 |\centering| 和 |center| 环境.当然在一般的文献中,居中一般用于摘要,你也可以用 |abstract| 环境.
	\end{itemize}
\end{frame}

\def\formto{ \raisebox{.18ex}{ $\to$ } }
\begin{frame}{另一个例子}
	\begin{block}{例子}
		\includegraphics[width = \textwidth]{images/Theorem example.pdf}
	\end{block}
	\begin{itemize}
		\item<+-> 粗体\formto
		      「公理 1」\formto
		      取消粗体\formto
		      「(自然单位)」\formto
		      粗体\formto
		      句点\formto
		      取消粗体\formto
		      楷体(斜体)\formto
		      「上帝曾言,」\formto
		      公式开始\formto
		      「$\hbar = c = 1$」\formto
		      公式结束\formto
		      句点;
		\item<+-> 不应该思考「这一小块内容应该是加粗、倾斜、字号几何」等问题;
		\item<+-> 这是「定理」\formto
		      它的名字是「勾股定理」\formto
		      它的内容是「设直角三角形……」\formto
		      把它们分别放在正确环境中.
        \item<+-> 如果写作的过程中会需要你不停调整格式,这意味着\CJKsout{你喜欢调整格式}你应该切换思考方式,写东西只需要一个框架,一切完工后再慢慢设计不迟.
	\end{itemize}
\end{frame}

\begin{frame}[fragile]{布局:浮动体不是\alert{不动体}}
	\begin{itemize}
		\item<+-> 动了吗?如动.\kern\ccwd ——古谚
		\item<+-> 主要的浮动体是 |figure| 和 |table| 环境:
			\begin{itemize}
				\item 标题用 |\caption|;
				\item 标签用 |\label|;
				\item 引用用 |\ref| 或者 |\autoref| (需要 \pkg{hyperref} 宏包);
			\end{itemize}
		\item<+-> 浮动体是会动的,\alert{故意的}.
			\begin{itemize}
				\item 治本型:不要用「上图下表」;
				\item 眼不见心不烦型:先写文章,全部写完后再插图片;
				\item 摆烂型:所有图片全部放最后面;
				\item 实用主义型:不就是要个编号吗?我手动加上去就是了:\pkg{caption} 宏包.(不是很推荐)
			\end{itemize}
		\item<+-> 归根结底就是,\LaTeX{}是不会急的,所以请你也别急.
	\end{itemize}
\end{frame}

\begin{frame}[fragile]{布局:参考文献}
	\begin{itemize}
		\item 我们一开始给出了 \BibTeX{} 的大致使用方式,更现代的方法可以用 |biber| 后端配合 \pkg{biblatex} 宏包;
		      \begin{texcode}[breaklines=true,moretexcs={\addbibresource,\printbibliography}]
    % 引言区
    \usepackage[backend=biber,style=gb7714-2015]{biblatex}
    \addbibresource[location=local]{cite.bib}
    % \begin{document} 之后
    \printbibliography[heading=bibliography,title=参考文献]
\end{texcode}
		      然后分多步编译:
		      \begin{texcode}
    xelatex jobname.tex
    biber jobname
    xelatex jobname.tex
    xelatex jobname.tex
\end{texcode}
	\end{itemize}
\end{frame}