\begin{texcode}[basicstyle=\tiny\ttfamily, breaklines = true, emph={[1]definition,enumerate,document}, emph={[2]article,amsmath,amsthm}, moretexcs={\mathcal,\maketitle,\subsection}]
\documentclass[11pt]{article}        % 使用 article 文档类
\usepackage{amsmath}
\usepackage{amsthm}                  % 关于定理环境的包
\newtheorem{definition}{Definition}  % 定理环境声明
\title{States and Observables}
\author{Jan van Neerven}
\date{2022}
\begin{document}
\maketitle                           % 显示你的标题
\section{%*\textbf{States and Observables in Classical Mechanics}*)}
We start by taking a brief look at ...
\subsection{%*\textbf{States}*)}
In classical mechanics, the state ...
\begin{definition}[States, pure states]
Let $(X,\mathcal{X})$ be a measurable space.
\begin{enumerate}
    \item A \emph{state} is ...
    \item A \emph{pure state} is ...
\end{enumerate}
\end{definition}
For a measurable set $B\in X$ , the number $\nu(B)$ is thought of as ``the probability that the state is described by a point in $B$''.
\end{document}
\end{texcode}