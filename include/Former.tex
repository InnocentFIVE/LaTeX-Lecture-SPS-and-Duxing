\sectionwithabstract{从零}[\NotoSerifCJKTC 萬物是藉著他造的;凡被造的,沒有一樣不是藉著他造的.\\\mbox{}\hfill ---\textsc{John 1:3}. \emph{The Word Became Flesh}%
]

\begin{frame}{实用主义者的自白}
	\pause 在一切一切开始之前……我们用最简洁的方式来介绍 \LaTeX{},我们将会:\pause
	\begin{itemize}
		\item<+-> 给出写出一篇(最基本的)文档的方法,将更具体的\LaTeX{}特性阐释后置;
		\item<+-> 也许你会说:“\alert{我还不知道\LaTeX{}怎么下载呢!}”别急,我们将会使用在线的 TeXPage;
		\item<+-> 在这个阶段,我说的话可能是有失偏颇的……但说得太多可能会引发初学者的畏难情绪,而这在一开始是没有必要的;
		\item<+-> 总而言之,先让它动起来再说.
	\end{itemize}
\end{frame}

\begin{frame}{简要的介绍}
    \begin{itemize}
        \item<+-> \LaTeX{} 是什么?
        \begin{itemize}
            \item<+-> 粗略来说,是 Leslie Lamport 对 \TeX{} 的一种封装;
            \item<+-> “排版公式很厉害且写论文 / 提交作业要用的”的排版系统;
            \item<+-> 从嘴里说出来时,\LaTeX{}(大概)是\TeX{}、\LaTeX{}内核、文档类、宏包的总称;
            \item<+-> \LaTeX{} 语法是数学家之间的交流方式.
        \end{itemize}
        \item<+-> \TeX{} 是什么?
        \begin{itemize}
            \item<+-> Donald~E.\ Knuth 和他的学生开发的排版文字和数学公式而开发的排版引擎;
            \item<+-> \LaTeX{} 的底层,真正负责排版的东西.
        \end{itemize}
    \end{itemize}
\end{frame}



\begin{frame}[t,fragile]{演示:\LaTeX{},启动!}
    按照惯例,我们先来给出我们的第一篇文档(还有最基本的中英混排的例子).\pause 
	\begin{texcode}[emph={[1]document}, emph={[2]article}]
    \documentclass{article}
    % 这里是导言区,% 用来表示注释
    \begin{document}
    Hello, world!
    \end{document}
\end{texcode}\pause 
    \begin{itemize}
        \item<+-> 其中,|\documentclass| 是一个控制序列,|{article}| 是其的一个必要参数,参数值是 |article|,表示调用 \pkg{article} 文档类.

        \item<+-> 导言区是用来设置整篇文档的格式的,他们往往不会决定文档中输出的\alert{内容}.
    
        \item<+-> |\begin{document}| 与 |\end{document}| 之间的是最基本的 |document| 环境,用来填写主要内容的地方.
\end{itemize}
\end{frame}

\begin{frame}[t,fragile]{演示:中文支持}
	\pause
	先上最新最热的配置:\CTeX{} 宏集 $+$ \XeLaTeX{} 程序.如果你不清楚这两个名词意味着什么,请看下面的代码:\pause
	\begin{texcode}[emph={[1]document}, emph={[2]ctexart}, breaklines = true]
    \documentclass{ctexart}
    % 使用 XeLaTeX 编译
    \begin{document}
    你好 :)
    \end{document}
\end{texcode}\pause
	\begin{itemize}
		\item<+-> \CTeX{} 宏集是面向中文排版的大框架,而并非是 \CTeX{} 套装,\pkg{ctexart} 是专为中文排版理过的 \pkg{article} 类;
		\item<+-> \XeLaTeX{} 总体说来是一个编译工具,在 TeXPage 里,我们可以看到 \LuaLaTeX{}、\pdfLaTeX{} 等类似物,但对中文排版来说,\XeLaTeX{} 是最合适的;
		\item<+-> 过于古典的方法可能会导致一些问题,比如「复制出乱码」.
	\end{itemize}
\end{frame}

\begin{frame}[fragile]{演示:数学公式(一)}
	\pause
	\begin{itemize}
		\item<+-> 当我们第一次使用 \LaTeX{} 时,大抵上都是为了输入数学公式.
		\item<+-> \LaTeX{} 上的数学环境可以分为两种:
			\begin{itemize}
				\item 行内模式(|$...$|):$\mathfrak L^2(G) \cong \bigoplus_{\pi \in \widehat{G}}\mathcal E_\pi$;
				\item 行间模式(|\[...\]| 或 |equation| 环境):
				      \[\mathcal B(\mathbb R) = \bigcup_{\xi<w_1}\symbf\Sigma_\xi^0(\mathbb R) = \bigcup_{\xi<w_1}\symbf\Delta_\xi^0(\mathbb R) = \bigcup_{\xi<w_1}\symbf\Pi_\xi^0(\mathbb R).\]
			\end{itemize}
		\item<+-> \pkg{amsmath} 宏包是科学文献写作的事实标准,最好始终调用:\onslide<+->
			\begin{texcode}[breaklines = true, emph={[1]document}, emph={[2]ctexart,amsmath}]
    \documentclass{ctexart}
    \usepackage{amsmath} % 用 \usepackage 调用宏包
    \begin{document}
        \[\hbar = c = 1.\]
    \end{document}
\end{texcode}
	\end{itemize}
\end{frame}

\begin{frame}[fragile]{演示:数学公式(二)}
	\pause
	\begin{itemize}
		\item<+-> 括号.|()|、|[]|、|\{\}|.自动控制大小是 |\left...|、|\right...|:
			\begin{columns}
				\begin{column}{.55\textwidth}
					\begin{texcode}[breaklines=true,basicstyle=\scriptsize\ttfamily]
        \[
        %*\textcolor{橙黄!70!朱红}{\string\left(}*)\frac{1}{3}%*\textcolor{橙黄!70!朱红}{\string\right]}*)
        +%*\textcolor{橙黄!70!朱红}{\string\left\string\{}*)\frac{1}{x}%*\textcolor{橙黄!70!朱红}{\string\right\string\}}*)^2
        \]
\end{texcode}
				\end{column}
                \onslide<+->
				\begin{column}{.4\textwidth}
					\[
						\left( \frac{1}{3} \right]
						+\left\{ \frac{1}{x} \right\|^2
					\]
				\end{column}
			\end{columns}
		\item<+-> 上下标(|^{...}|、|_{...}|)、分式(|\frac{分子}{分母}|)、根式(|\sqrt[次数]{内容}|):
			\begin{columns}
				\begin{column}{.55\textwidth}
					\begin{texcode}[breaklines=true,basicstyle=\scriptsize\ttfamily]
        \[
        \int_0^1 x^7 dx = 
        \frac{1}{8} = 
        %*\textcolor{橙黄}{\string\left(}*) \frac{\sqrt[3]{8}}{2} %*\textcolor{橙黄}{\string\right)}*) ^3.
        \]
\end{texcode}

				\end{column}
                \onslide<+->
				\begin{column}{.4\textwidth}
					\[\int_0^1 x^7 dx = \frac{1}{8} = \left( \frac{\sqrt[3]{8}}{2} \right) ^3.\]
				\end{column}
			\end{columns}
	\end{itemize}
\end{frame}

\begin{frame}[fragile]{演示:数学公式(三)}
	\begin{itemize}
		\item<+-> 矩阵.在调用 \pkg{amsmath} 宏包的情形下,使用 |pmatrix| 环境,其中 |&| 是分列,|\\| 是换行:
	\end{itemize}
	\begin{columns}
		\begin{column}{.45\textwidth}
			\begin{texcode}[gobble=4, emph={[2]pmatrix}]
        \begin{pmatrix}
            1 & 0 & 0  & 0 \\
            0 & 0 & −1 & 0 \\
            0 & 1 & 0  & 0 \\
            0 & 0 & 0  & 0 \\
        \end{pmatrix}.
\end{texcode}
		\end{column}
        \onslide<+->
		\begin{column}{.45\textwidth}
			\[
				\begin{pmatrix}
					1 & 0 & 0  & 0 \\
					0 & 0 & −1 & 0 \\
					0 & 1 & 0  & 0 \\
					0 & 0 & 0  & 0 \\
				\end{pmatrix}.
			\]
		\end{column}
	\end{columns}
\end{frame}

\begin{frame}[fragile]{演示:组织文本(一)}
	\begin{itemize}
		\item<+-> 节与段:|\section{...}|;小小节:|\subsection{...}|、段落:|\paragraph{...}|;
		\item<+-> 定理与证明:\onslide<+->
			\begin{texcode}[emph={[2]amsthm},emph={[1]theorem,document,proof}]
\usepackage{amsthm}
\newtheorem{theorem}{定理}
\begin{document}
    \begin{theorem}[Riemann假设]
        Riemann $\zeta$函数的非平凡零点实部均为$1/2$。
    \end{theorem}

    \begin{proof}
        因为这是假设,我们不妨假设显然成立。
    \end{proof}
\end{document}
\end{texcode}
	\end{itemize}
\end{frame}

\begin{frame}[fragile]{演示:组织文本(二)}
	\begin{itemize}
		\item 列举环境:|itemize|、|enumerate|、|description|;\onslide<+->
		      \begin{texcode}[breaklines = true, emph={[1]enumerate,itemize,description}, moretexcs={\arabic*}]
% \usepackage{enumitem} 
% 处理 enumerate 环境编号格式
\begin{enumerate}[label=(\arabic*).]
\item Mark~Srednicki;
\item A.~Zee;
    \begin{itemize}
        \item M.~Peskin and D.~Schroede;
        \item S.~Weinberg;
    \end{itemize}
    \begin{description}
        \item [L.~Ryder] Quantum Field Theory;
        \item [David Tong] QFT;
    \end{description}
\end{enumerate}
\end{texcode}
	\end{itemize}
\end{frame}

\begin{frame}[t,fragile]{演示:图片}
	\begin{columns}
		\begin{column}{.58\textwidth}
			\begin{texcode}[breaklines=true, moretexcs={\includegraphics,\autoref}, emph={[1]figure}, emph={[2]hyperref,graphicx}]
% \usepackage{graphicx} 
% 用来 \includegraphics
% \usepackage{hyperref} 
% 超链接
\begin{figure}
    \centering
    \includegraphics[width=4cm]{images/li-a-ling.jpg}
    \caption{李阿玲}
    \label{%*\textcolor{蔚蓝!80!黑色}{fig:li-a-ling}*)}
\end{figure}
尽可能用「如 \autoref{%*\textcolor{蔚蓝!80!黑色}{fig:li-a-ling}*)}
所示」这种语言而不是「见上图、下表」。
\end{texcode}
		\end{column}
        \pause
		\begin{column}{.4\textwidth}
			\footnotesize
			\begin{figure}
				\centering
				\includegraphics[width=4cm]{images/li-a-ling.jpg}
				\caption{\figurename\thefigure: 李阿玲}
				\label{fig:li-a-ling}
			\end{figure}
			尽可能用「如\figurename\autoref{fig:li-a-ling} 所示」这种语言而不是「见上图、下表」.
		\end{column}
	\end{columns}
	\nonumberfootnote{图片来源:\link{https://www.zhihu.com/people/li-a-ling}}
\end{frame}

\begin{frame}{演示:表格}
	\begin{columns}
		\begin{column}{.6\textwidth}
			\begin{texcode}[basicstyle=\scriptsize\ttfamily, breaklines=true, moretexcs={\toprule,\midrule,\bottomrule,\qty}, emph={[1]table,tblr},emph={[2]tabularray,siunitx,booktabs}]
% \usepackage{tabularray}   % 个人喜好
% \usepackage{siunitx}      % 排版单位
% \UseTblrLibrary{booktabs} % 三线表
\begin{table}
\centering
\caption{...}
\label{%*\textcolor{蔚蓝!80!黑色}{tab:masses-particles}*)}
\begin{tblr}{lll}
    \toprule
    Particle    & Mass               \\
    \midrule
    electron    & \qty{.5}{MeV}      \\
    ...         & ...                \\
    Higgs Boson & \qty{125}{GeV}     \\
    \bottomrule
\end{tblr}
\end{table}
\end{texcode}
		\end{column}
        \pause
		\begin{column}{.4\textwidth}
			\begin{table}
				\footnotesize
				\centering
				\caption{\tablename\thetable: The rough masses of some elementary \emph{(and not so elementary)} particles}
				\begin{tblr}{lll}
					\toprule
					Particle        & Mass                                                       \\
					\midrule
					electron        & \qty{.5}{MeV}                                              \\
					Muon            & \qty{100}{MeV}                                             \\
					Pions           & \qty{140}{MeV}                                             \\
					Proton, Neutron & \qty{1}{GeV}                                               \\
					Tau             & \qty{2}{GeV}                                               \\
					W, Z Bosons     & \qtyrange[range-phrase=--,range-units=single]{80}{90}{GeV} \\
					Higgs Boson     & \qty{125}{GeV}                                             \\
					\bottomrule
				\end{tblr}
			\end{table}
		\end{column}
	\end{columns}
	\nonumberfootnote{数据来源:\textsc{David Tong.}~\emph{Quantum Field Theory}. \link{https://www.damtp.cam.ac.uk/user/tong/qft/qft.pdf}}
\end{frame}

\begin{frame}[fragile]{演示:参考文献(\BibTeX{})}
	\begin{itemize}
		\item<1-> 先声明:\BibTeX{}是一个古典的方法,在这里使用它只是容易演示且资料丰富;
		\item<1-> 先要有一个 |.bib| 文件;
		\item<1-> |\bibliographystyle{...}|、|\cite{...}|、|\bibliography{...}|;
		\item<2-> 文献的\BibTeX{}怎么找?(自己写?)
		\item<3-> 想要输出所有文献但懒得引用?(|\nocite{*}|).
		\item<3-> \alert{要编译多次},否则就等着爆 \textbf{[??]} 吧.
	\end{itemize}\onslide<4->
	\begin{texcode}[morekeywords={\cite,\bibliographystyle,\bibliography,\nocite}]
	\cite{%*\textcolor{emph1}{\emph{simon2015operator}}*)}
	\bibliographystyle{alpha}
	\nocite{*}
	\bibliography{cite}
\end{texcode}
\end{frame}