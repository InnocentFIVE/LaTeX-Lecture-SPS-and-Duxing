% \begin{frame}[noframenumbering]
% 	\thispagestyle{empty}
% 	\centering
% 	\section{数学公式}
% 	\vskip-.5\baselineskip
% 	\begin{minipage}[t][0pt][t]{22em}\footnotesize
% 		很多常见的公式服务依赖于一些轻量级渲染引擎,比如MathJax、KaTeX.但是它们实际上使用的是\LaTeX{}语法变种,也就是说并没有使用\LaTeX{}后端.所以不要期望有完全一致的输出.
% 	\end{minipage}
% 	\vskip-1.5\baselineskip
% 	\vskip-\parskip
% 	\vskip-6pt
% \end{frame}

\sectionwithabstract{数学公式}[%
	很多常见的公式服务依赖于一些轻量级渲染引擎,比如MathJax、KaTeX.但是它们实际上使用的是\LaTeX{}语法变种,也就是说并没有使用\LaTeX{}后端.所以不要期望有完全一致的输出.%
]


\begin{frame}[fragile]
	\frametitle{如何在数学环境下输入中文}
	\begin{itemize}
		\item<+-> 想要在 \LaTeX{} 输入中文,直接输入是做不到的,最有性价比的方法是使用 |\text{...}|,这需要调用 \pkg{amsmath} 宏包:
			|\[\hbar=1,\quad \text{真的}.\]|
			\[\hbar=1,\quad \text{真的}.\]
		\item<+-> 数学环境中的空格会被忽略,但 |\text{...}| 中的不会.
			\begin{itemize}
				\item |$A \text{开集}$|:$A\text{开集}$;
				\item |$A\text{ 开集}$|:$A\text{ 开集}$.
			\end{itemize}
	\end{itemize}
\end{frame}

\begin{frame}[fragile]
	\frametitle{数学环境的结构}
	\begin{itemize}
		\item 数学环境的结构分为以下几种:
		      \begin{itemize}
			      \item 原子.如$A$, $\Delta $, $f$之类;
			      \item 开闭结构.如$($, $]$, $\langle$之类;
			      \item 算子.如$\sin$, $\tan$之类,最主要的特征是\alert{正体};
			      \item 大算子.如$\sum$, $\bigoplus$, $\iiiint$之类;
			      \item 标点.基本的逗号,分号和句号;
			      \item 关系符或者二元算符.$<$, $+$, $\to$之类.
		      \end{itemize}
	\end{itemize}
\end{frame}

\begin{frame}[fragile]{数学环境的细节(一)}
	\begin{itemize}
		\item<+-> 括号.大括号在 \LaTeX{} 中起分组作用,输入应用 |\{|, |\}|,绝对值可以用 \lstinline[style=style@inline]'|...|',范数可以用 \lstinline[style=style@inline]'\|...\|'.
			\[\{A\} = \|\Psi \| + \vert x\vert.\]
		\item<+-> 大大大大大括号.
			\begin{itemize}
				\item 用 |\left|, |\right| 是最省事的方法:
				      \begin{columns}
					      \begin{column}{.58\textwidth}
						      \begin{texcode}[moretexcs={\geqslant}]
        %*\textcolor{橙黄!70!朱红}{\string\left(}*)1, \sum_{n\geqslant 1}
        \frac{1}{n^2} %*\textcolor{橙黄!70!朱红}{\string\right]}*)
\end{texcode}
					      \end{column}
					      \begin{column}{.35\textwidth}
						      \[\left(1, \sum_{n\geqslant 1} \frac{1}{n^2} \right] \]
					      \end{column}
				      \end{columns}
				\item 手动解决方案:|\big|, |\Big|, |\bigg|, |\Bigg| 四挡.
			\end{itemize}
	\end{itemize}
\end{frame}

\begin{frame}[fragile]{数学环境的细节(二)}
	\begin{itemize}
		\item<+-> 张量.可以用 |tensor| 宏包.|\tensor{\Lambda }{^\mu _\sigma }| 而不是 |\Lambda ^\mu _\sigma|:
			\[\tensor{\Lambda }{^\mu _\sigma }\neq \Lambda ^\mu _\sigma.\]
		\item<+-> 超大根式可以用 |^{p/q}| 之类描述:
			\[
				\sqrt[3]{\sum_{i=1}^N\frac{1}{(p_i+p_i')^2-m^2}}\qquad \Bigl(\sum_{i=1}^N\frac{1}{(p_i+p_i')^2-m^2}\Bigr)^{1/3}.
			\]
		\item<+-> 分式(|\frac{分子}{分母}|).在行间或角标请使用 |a/b|.你也不想遇到这种吧:\onslide<+->
			\begin{gather*}
				\Bigl(x+\frac{\mathrm e}{\mathrm e-1}\Bigr)\mathrm e^{\frac{x^{\frac{\mathrm e}{\mathrm e-1}}}{\mathrm e-1}},\quad
				\overline{E} = \frac{\int_0^\infty E \exp (-E/(k_{\mathrm  B}T))\d E}{\int_0^\infty \exp (-E/(k_{\mathrm  B}T))\d E} = k_{\mathrm B}T.
			\end{gather*}
	\end{itemize}
\end{frame}

\begin{frame}[standout]
	\[\Bigl(x+\frac{\mathrm e}{\mathrm e-1}\Bigr)\exp \Bigl( \frac{x^{\mathrm e / (\mathrm e-1)}}{\mathrm e-1}\Bigr).\]
	\medskip
	\[\overline{E} = \int_0^\infty E \exp \Bigl(-\frac{E}{k_{\mathrm B}T}\Bigr)\d E\biggm/\!\int_0^\infty \exp \Bigl(-\frac{E}{k_{\mathrm B}T}\Bigr)\d E = k_{\mathrm B}T.\]
\end{frame}

\begin{frame}[fragile]{数学环境的细节(三)}
	\begin{itemize}
		\item<+-> 数学环境下的空格(还记得吗?数学环境下一般空格无效)\footnote{值得一提的是,在2020-10-01以后,或者调用 \pkg{amsmath} 后,也可以在文本模式下使用这些命令.}.细(|\,|)、中(|\:|, |\>|)、粗(|\;|):|\quad|, |\qquad|:
			\begin{center}
				\begin{tblr}{ll}
					\fakeverb{\,}     & $D\,X     $ \\
					\fakeverb{\>}     & $D\>X     $ \\
					\fakeverb{\;}     & $D\;X     $ \\
					\fakeverb{\quad}  & $D\quad X $ \\
					\fakeverb{\qquad} & $D\qquad X$ \\
				\end{tblr}
			\end{center}
        \item<+-> 不建议在行内插入过于复杂的公式.
	\end{itemize}
\end{frame}

\begin{frame}[fragile]{数学字体的切换}
	\begin{itemize}
		\item 数学字体的切换.最好选择 \pkg{amsfonts} 宏包,更现代的选择是 \pkg{unicode-math} 宏包.
		      \begin{itemize}
			      \item<1-> 主要处理:|\math...{}| 和 |\sym...{}|(\pkg{unicode-math} 宏包);
			      \item<2-> 粗体:|\mathbf|, |(\sym)bfit|, |\boldsymbol|;
			      \item<2-> 花体:|\mathcal|, |\mathscr|(需要一些花体宏包,如 \pkg{mathalpha});
			      \item<2-> 哥特体和黑板粗体:|\mathfrak|, |\mathbb|.
			      \item<2-> 但是,数学字体这些东西并不是从天上掉下来的.正如前文所言,数学环境调用的字体与正文\alert{不同},因此不受正文字体环境影响.
			      \item<3-> \fontspec{Times New Roman}{Times New Roman}:使用 \pkg{newtx} 宏包.
		      \end{itemize}
	\end{itemize}
\end{frame}

% \begin{frame}[fragile]
% 	\frametitle{数学环境的细节(二)}
% 	\begin{itemize}
% 		\item 矩阵.在调用 \pkg{amsmath} 宏包的情形下,使用 |pmatrix| 环境,其中 |&| 是分列,|\\| 是换行:
% 	\end{itemize}
% 	\begin{columns}
% 		\begin{column}{.45\textwidth}
% 			\begin{texcode}[gobble=4, emph={[2]pmatrix}]
        \begin{pmatrix}
            1 & 0 & 0  & 0 \\
            0 & 0 & −1 & 0 \\
            0 & 1 & 0  & 0 \\
            0 & 0 & 0  & 0 \\
        \end{pmatrix}.
\end{texcode}
% 		\end{column}
% 		\begin{column}{.45\textwidth}
% 			\[
% 				\begin{pmatrix}
% 					1 & 0 & 0  & 0 \\
% 					0 & 0 & −1 & 0 \\
% 					0 & 1 & 0  & 0 \\
% 					0 & 0 & 0  & 0 \\
% 				\end{pmatrix}.
% 			\]
% 		\end{column}
% 	\end{columns}
% \end{frame}



\begin{frame}[fragile]{更高更妙的物理(一)}
	\begin{itemize}
		\item<+-> Dirac 符号(\pkg{physics2} 宏包):
			\[A^\dagger = \sum_i \lambda _i^*\ketbra{u_i}{v_i}.\]
			(有人曾经用$\vert \phi >$来表示$\ket{\phi }$,我不说是谁)
		\item <+->合理的单位制(\pkg{siunitx} 宏包).
		      \begin{itemize}
			      \item |\hbar=1.0545\times 10^{−34}J\cdot s|: $\hbar=1.0545\times 10^{−34}J\cdot s$.
			      \item |\hbar=\qty[options...]{1.0545e-34}{J.s}|: $\hbar=\qty[inter-unit-product = \ensuremath{{}\cdot{}}]{1.0545e-34}{J.s}$.
		      \end{itemize}
		\item<+-> 微分算子分不清?是 $\mathscr D$, $d$ 还是 $\mathbb d$ 呢?
			\begin{itemize}
				\item 古老的国标 GB/T~3012.11--93 成为推荐性,用什么都是可以哒!
				\item \pkg{fixdif} 宏包,更方便考虑间距: $\int^d \frac{\d D(d)}{\d d}\d d = D(d)+C$.
			\end{itemize}
	\end{itemize}
\end{frame}

\begin{frame}{更高更妙的物理(二)}
	\begin{itemize}
		\item<+-> Feynman 图(\pkg{tikz-feynman} 宏包).
	\end{itemize}
	\begin{figure}
		\centering
		\begin{tikzpicture}
			\begin{feynman}
				\vertex                         (i1) {\(e^{-}\)};
				\vertex [below right=2cm of i1] (a);
				\vertex [below=1.5cm of a]      (b);
				\vertex [below left=1.5cm of b] (f1){\(e^{+}\)};

				\vertex [right=3cm of i1]       (i2) {\(e^{+}\)};
				\vertex [right=3cm of f1]       (f2) {\(e^{-}\)};


				\diagram* {
				(i1) -- [fermion] (a),% e-
				(a) -- [boson, edge label'=\(p\)](b),
				(b) -- [fermion](f1),% e+

				(i2) -- [fermion] (b),
				(a) -- [fermion] (f2)
				};
			\end{feynman}
		\end{tikzpicture}
		\caption{\addfontfeatures{RawFeature=-case}\figurename\thefigure: u-channel}
	\end{figure}
\end{frame}


\begin{frame}[standout]
	\small
	\fakeverb{bm = \boldsymbol{...}}、\fakeverb{cal = \mathcal{...}},以此类推.

	\begin{tblr}{XXXX}
		\fakeverb{rm}  & \fakeverb{bf}       & \fakeverb{sf}   & \fakeverb{bm + sf}   \\
		               & \fakeverb{bm}       & \fakeverb{-}    & \fakeverb{-}         \\
		\fakeverb{cal} & \fakeverb{bm + cal} & \fakeverb{bb}   & \fakeverb{-}         \\
		\fakeverb{scr} & \fakeverb{-}        & \fakeverb{frak} & \fakeverb{bm + frak} \\
	\end{tblr}
	\LARGE
	\[
		\mathfontlist{D}\qquad \mathfontlist{x}
	\]
	\nonumberfootnote{字体:Minion Math, Euler, Garamond Math.}
\end{frame}

\begin{frame}[fragile]{多行公式与公式对齐(一)}
	|align| 环境或 |gather| 环境.在调用 \pkg{amsmath} 宏包的情形下,也可以用 |aligned| 和 |gathered|.同矩阵一样,|&| 分列,|\\| 对齐,用 |\tag| 和 |\notag| 控制标号.\pause
	\begin{columns}
		\begin{column}{.5\textwidth}
			\vskip.5\ccwd
			\begin{texcode}[basicstyle=\scriptsize\ttfamily, emph={[1]align}, moretexcs={\notag,\tag}]
  \begin{align}
      \hbar = c & = 1          \\
      \notag    & = 137\alpha  \\
      \tag{42}  & = \hat\delta.
  \end{align}
\end{texcode}
		\end{column}
		\pause
		\begin{column}{.42\textwidth}
			\begin{align}
				\hbar  = c & = 1           \\
				\notag     & = 137 \alpha  \\
				\tag{42}   & = \hat\delta.
			\end{align}
		\end{column}
	\end{columns}
	\pause
	想要去掉所有标号可以用\CJKsout{共轭算符} |*| 环境:
	\begin{columns}
		\begin{column}{.5\textwidth}
			\vskip.5\ccwd
			\begin{texcode}[basicstyle=\scriptsize\ttfamily, emph={[1]gather,gather*}]
  \begin{gather*}
  p(E = n\hbar\omega) =
  \frac{e^{−\beta n\hbar\omega}}{Z}; \\
  Z = \frac{1}{1−e^{−\beta\hbar\omega}}.
  \end{gather*}
\end{texcode}
		\end{column}
		\pause
		\begin{column}{.42\textwidth}
			\begin{gather*}
				p(E = n\hbar\omega)
				= \frac{e^{−\beta n\hbar\omega}}{Z}; \\
				Z = \frac{1}{1 − e^{−\beta\hbar\omega}}.
			\end{gather*}
		\end{column}
	\end{columns}
\end{frame}

\begin{frame}[fragile]{多行公式与公式对齐(二)}
	当多行公式只需一个标号时,可以用 |equation| 套上 |aligned| 或 |gathered| 环境解决:
	\begin{columns}
		\begin{column}{.6\textwidth}
			\vskip.5\ccwd
			\begin{texcode}[basicstyle=\scriptsize\ttfamily, emph={[1]equation,aligned}, moretexcs={\notag}]
  \begin{equation}
  \begin{aligned}
      \partial_t p & = -\partial_x H, \\
      \d_t x       & = \partial_p H.
  \end{aligned}
  \end{equation}
\end{texcode}
		\end{column}\pause
		\begin{column}{.35\textwidth}
			\color{black}
			\begin{equation}
				\begin{aligned}
					\partial_t p & = -\partial_{ x}H, \\
					\d_t x       & = \partial_{ p}H.
				\end{aligned}
			\end{equation}
		\end{column}
	\end{columns}
	\pause
	请不要使用 |\[...\] \[...\] \[...\]|!更多需求可以使用 |alignat| 和 |flalign| 等,总有一款适合你.
\end{frame}

\begin{frame}[fragile]{多行公式与公式对齐(三)}
	如果要对很多子公式进行主–子编号, 可以用 |subequations| 环境搭配 |align| 和 |gather| 使用:\pause
	\begin{texcode}[basicstyle=\scriptsize\ttfamily, emph={[1]subequations,gather}, moretexcs={\mathcal,\coloneqq,\text,\set,\given}]
    \begin{subequations}
        \begin{gather}
            \mathcal{H}_{\text{ac}} \coloneqq
            \set{x\in \mathcal{H}\given \mu_x \ll m}, \\
            %*\textcolor{comment}{......}*)
        \end{gather}
    \end{subequations}
\end{texcode}\pause
	\begin{subequations}
		\begin{gather}
			\symcal{H}_{\textup {ac}}
			\coloneqq \set{x\in \symcal{H}\given \mu _x \ll m},                                                                \\
			\symcal{H}_{\textup {sc}}
			\coloneqq \set{x\in \symcal{H}\given \textup{\(\mu _{x,x} \mathrel\bot m\) but \(\mu_{x,x}(\{\textup{pt}\})=0\)}}, \\
			\symcal{H}_{\textup {pp}}
			\coloneqq \set{x\in \symcal{H}\given \textup{\(\lambda \mapsto \braket[1]{E_\lambda x,x}\) is discrete}},                                      \\
			\symcal{H} =\symcal{H}_{\textup {ac}}\oplus\symcal{H}_{\textup {sc}}\oplus\symcal{H}_{\textup {pp}}.
		\end{gather}
	\end{subequations}
\end{frame}

\begin{frame}[fragile]{符号}
	\begin{itemize}
		\item 明确一点,大多数(抽象的)符号都只是编码,也即:\pause
		      \begin{itemize}
			      \item |U+1D510|:$\symfrak M$;
			      \item |U+02B32|:$\leftarrowonoplus$;
			      \item |U+060A8|:您;
			      \item ……
		      \end{itemize}
		      具体显示效果来源于\alert{字体},不同的字体对同样的符号显示效果可能是不一样的.\pause
		\item 因此「符号」不可能只用一个按钮或者一个命令就可以从天而降:需要字体;需要 \LaTeX{} 调用字体;需要语义化的命令而不是直接输入编码得到符号……
	\end{itemize}
\end{frame}

\begin{frame}[fragile]{如何寻找符号?}
	\begin{itemize}
		\item<1-> 由 \TeX{} 宏包所定义的符号:\emph{The Comprehensive \LaTeX~Symbol List}~\link{https://tug.ctan.org/info/symbols/comprehensive/symbols-a4.pdf};
		\item<2-> \pkg{unicode-math} 宏包使用者:\emph{万有Unimath}~\link{https://mirror-hk.koddos.net/CTAN/macros/unicodetex/latex/unicode-math/unimath-symbols.pdf};
		\item<3-> 仅仅知道形状?\emph{Detexify}~\link{http://detexify.kirelabs.org/classify.html};
		\item<4-> 我要苹果商标 \faApple!这个字形没有收录在Unicode标准中,换言之,只有特殊的字体才会有(亦即,在字体的私有区中制作),可见 \pkg{fontawesome5} 或者 \pkg{simpleicons} 宏包;
		\item<5-> emoji(\twemoji[height = .88\ccwd]{mahjong})?emoji 也有对应的 Unicode 规范,但\hologo{XeTeX} 对彩色字体支持性不高,可以用 Lua\kern-.2ex\TeX{} 或者 \pkg{twemoji} 宏包;
		\item<6-> 用 |\%|、|\$|、|\&|、|\textbackslash| 得到~\%、\$、\&、\textbackslash{};
	\end{itemize}
\end{frame}